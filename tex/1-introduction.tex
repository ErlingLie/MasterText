%!TEX root = ../Thesis.tex
\chapter{Introduction}\label{cha:introduction}
%
This introductory chapter motivates the work by discussing the various applications of cooperating \glspl{auv}. The formation path-following problem is motivated, and limitations posed on existing methods by the nonlinear equations of motion are discussed. The thesis problem is further motivated by briefly discussing the features of the hand-position controller and its possible applications. Then, a literature study on formation control methods for double-integrator systems is presented. These methods are viable candidates to be combined with the hand-position controller. The main contributions are listed and the chapter is concluded with an outline of the rest of the thesis.


\section{Motivation}\label{sec:motivation}
\glspl{auv} have become increasingly important for ocean research and exploration. \glspl{auv} are able to perform tasks in harsh and remote environments that may be too dangerous or difficult for human divers, such as collecting data on ocean temperatures, salinity, and currents, mapping the ocean floor, and conducting underwater inspections for the oil and gas industry. They have also been used for studying marine biology and geosciences \citep{das_data-driven_2015, wynn_autonomous_2014}. The use of \glspl{auv} has the potential to significantly improve our understanding of the ocean and its processes, as well as aid in the development of sustainable ocean practices. Unlike \glspl{rov}, \glspl{auv} are not tethered to the research vessel they are deployed from, which enables opportunities for explorations in areas that were previously inaccessible, such as beneath the ice in polar regions \citep{dowdeswell_autonomous_2008}.

The cooperation of multiple \glspl{auv} can enhance their capabilities and enable them to perform tasks that are difficult or impossible for a single \gls{auv} to accomplish. Cooperation among \glspl{auv} can increase mission efficiency, allows the exploration of larger areas, and provide redundancy in case of system failures. One example is the use of fleets of \glspl{auv} for oceanographic studies as mobile sensor networks \citep{leonard_collective_2007}.  Other applications include the use of multiple \glspl{auv} for the inspection of underwater structures or pipelines, in which each vehicle can be equipped with sensors and work together to cover a larger area more quickly. 

Our work concerns the multi-agent formation path-following problem. The vehicles are controlled to follow a desired path while keeping a predefined formation. The path can be preplanned or it can be provided by a higher-level control layer. Formation path following can be achieved using numerous different control strategies, including leader-follower approaches, where one \gls{auv} acts as a leader and the others follow its trajectory \citep{soorki_robust_2011, cui_leaderfollower_2010, wang_leader-follower_2009}, distributed path-following approaches using consensus algorithms \citep{skjetne_nonlinear_2002,ghabcheloo_coordinated_2006, borhaug_formation_2006}, and behavioral approaches that define the behaviors each \gls{auv} should exhibit to achieve a desired formation \citep{monteiro_dynamical_2002, balch_behavior-based_1998}.

What makes formation path following with \glspl{auv} especially complex compared to ground vehicles are the nonlinear underactuated dynamics. The dynamic constraints must be taken into account when designing formation-control algorithms for the systems to remain stable. The nonlinear and underactuated dynamics make many existing formation-control algorithms developed for other types of vehicles impossible or difficult to apply to \glspl{auv}. A promising approach is to use a hand-position input-output linearizing controller to transform the nonlinear equations into kinematic double-integrator systems \citep{matous_trajectory_2023, paliotta_trajectory_2019}, which may enable the application of a varied number of control strategies that could otherwise not be applied directly to \glspl{auv}. \cite{restrepo_tracking--formation_2022} successfully applied an edge-agreement-based distributed formation control law to 3-\gls{dof} \glspl{asv}, using the hand-position to transfer the \gls{asv} models into linear double-integrator dynamics. In this thesis, we will further explore formation control algorithms for multi-agent systems with double-integrator dynamics. We first present a literature study of control methods and then develop a formation control algorithm for double-integrator systems based on the \gls{nsb} control method.

\vspace*{-3mm}
\section{Literature review}\label{sec:literature_review}
This section presents a literature study on control methods for multi-agent systems with double-integrator dynamics. The methods can be grouped into three main categories: consensus-based methods, rigidity-based methods, and \gls{nsb} methods.

Various consensus algorithms have been proposed to solve the formation-keeping problem in double-integrator systems. The consensus problem is a problem in which multiple agents must coordinate to reach a common value in some information state. The information state can be for instance position, velocity, or a path-progress parameter. The formation path-following problem can be considered a special case of the consensus problem with the following objective:
\begin{equation}
    \lim_{t\rightarrow\infty} \mathbf{p}_i(t) - \mathbf{p}_j(t) - \mathbf{d}_{ij} = \mathbf{0},\quad \lim_{t\rightarrow\infty} \mathbf{v}_i(t) - \mathbf{v}_j(t) = \mathbf{0},
\end{equation}
where $\mathbf{p}_i$ and $\mathbf{v}_i$ are the position and velocity vectors for vehicle $i$ and $d_{ij}$ is the desired relative displacement between vehicle $i$ and $j$. For double-integrator systems, the general consensus controller will take the form
\begin{subequations}\label{eq:first_order_consensus}
\begin{align}
    \dot{\mathbf{p}}_i &= \mathbf{v}_i,\\
    \dot{\mathbf{v}}_i &= \bm{\mu}_i\\
    \bm{\mu}_i &= -k_1 \sum_{j\in\mathcal{N}_i}(\mathbf{p}_i-\mathbf{p}_j - \mathbf{d}_{ij}) -k_2 \sum_{j \in  \mathcal{N}_i}( \mathbf{v}_i - \mathbf{v}_j),
\end{align}
\end{subequations}
where $k_1$ and $k_2$ are constant controller gains and $\mathcal{N}_i$ is the set of all neighboring agents of agent $i$. Specific consensus algorithms can typically take slightly different forms depending on the goal, but will generally keep terms based on the difference in position and velocity with neighboring vehicles. A review of general consensus problems is found in \cite{ren_survey_2005}.

\cite{restrepo_tracking--formation_2022} developed a consensus control law that works directly with \glspl{asv} and \glspl{auv} equipped with the hand-position controller moving in the horizontal plane. It combines techniques from integrator backstepping, \glspl{blf}, and sliding-mode-like switching control. While the paper primarily addresses the tracking of an external target vehicle, it can easily be adapted to track a predefined trajectory using a virtual target. The method can be applied to path following by employing a leader-follower scheme, where the leader controls the progress along the path while the other vehicles act as followers. Moreover, in addition to addressing formation keeping and target tracking, the method explicitly tackles collision avoidance and the maintenance of a maximum communication range. The communication between neighboring agents is assumed to be undirected, allowing bidirectional information flow.\looseness=-1

Other consensus algorithms for control of multi-agent systems subject to double-integrator dynamics share many commonalities with each other. For instance, in \cite{miao_formation_2019}, a consensus algorithm is proposed in which a controller utilizes the gradient of a potential field to maintain communication radius, resembling the concept of \glspl{blf} in \cite{restrepo_tracking--formation_2022}. Despite using different notations, the control methods in both papers exhibit clear similarities. Furthermore, in \cite{montanez-molina_formation_2022}, a consensus algorithm is presented for formations with directed communication, employing a backstepping-like controller. Additionally, \cite{girejko_leader-following_2019} introduces a leader-follower tracking consensus controller with a virtual leader. In \cite{mohammadi_leader-following_2021}, a leader-follower tracking controller is proposed, where positions and velocities of the leader and other agents are estimated instead of being precisely known. Similar to \cite{restrepo_tracking--formation_2022}, a switching controller is applied to handle estimate uncertainties.

A different category of formation control methods is rigidity-based methods. \cite{cai_adaptive_2015} proposed a graph rigidity-based adaptive formation control law for vehicles moving in the plane. The robots' dynamics are modeled using Euler-Lagrange-like equations of motion and simplified using the hand position as the system output. The backstepping-based control law with inter-agent distances as controlled variables ensures asymptotic convergence to the desired formation without global position measurements. The rigid graph theory ensures the uniqueness of the formation shape under distance constraints. Because the method is adaptive, it allows for parametric uncertainty. \cite{li_adaptive_2021} formulate a similar method, but based on a bearing-rigid graph instead of a distance-rigid graph. They argue that bearing-only-based control methods have strong appeal due to the natural connection with vision-based control problems.

A third category of formation-control methods is the \gls{nsb} method. The method has so far been developed for multi-agent control of vehicles following single-integrator dynamics, but, as will be shown throughout this thesis, the method can be modified to work with double-integrator dynamics instead. The method was developed for 3-\gls{dof} \glspl{asv} in \cite{arrichiello_formation_2006}, and has been applied in various other works \citep{pereda_towards_2011, arrichiello_cooperative_2010, eek_formation_2021}. It was extended for \glspl{auv} moving in 6-\gls{dof} in \cite{matous_singularity-free_2022}. The main idea is to formulate a prioritized hierarchy of tasks the vehicle should follow in order to exhibit the desired behavior. The desired velocities generated by the lower-priority tasks are projected into the null space of the higher-priority tasks so that they are only satisfied within the subspace where they do not conflict.

\section{Contributions}\label{sec:contributions}
This thesis presents a novel control method for the formation path-following problem with \glspl{auv}. The main contributions can be summarized as follows:
\begin{itemize}
    \item A literature review on multi-agent control methods for systems following double-integrator dynamics.
    \item An extended second-order \gls{nsb} control method that works directly with double-integrator dynamics.
    \item An extension to the specific behavioral tasks developed in \cite{matous_singularity-free_2022} to work at the acceleration level to be compatible with the second-order \gls{nsb} method.
    \item A closed-loop stability analysis of the joint formation-control and path-following system with the extended \gls{nsb} controller.
    \item A novel approach to reformulating the \gls{nsb} algorithm into a distributed method.
    \item A closed-loop stability analysis of the proposed distributed \gls{nsb} method.
    \item An extensive MATLAB simulation study of the centralized and decentralized second-order \gls{nsb} method.
    \item A submitted conference paper to the 62nd IEEE Conference on Decision and Control with the initial results of the centralized second-order \gls{nsb} method \citep{lie_formation_2023}.
\end{itemize}


In this work, we develop an extended \gls{nsb} method for vehicles with double-integrator dynamics and propose an algorithm that uses a \gls{soclik} equation to control the task variables through acceleration inputs. The procedure is inspired by robotic manipulators, where second-order methods are more common,  due to the inherent second-order dynamics of mechanical systems \citep{siciliano_differential_2009, chiaverini_kinematically_2008}. Although existing \gls{nsb} methods are developed for first-order systems, \gls{auv} dynamics are inherently second-order. Therefore, any first-order solution is necessarily perturbed by the dynamics of the maneuvering controller. In contrast, our formulation handles the second-order dynamics directly in the task space as interpretable spring-damper systems.

We apply the 3D hand position method proposed in \cite{matous_trajectory_2023}, which transforms the underactuated six-degrees-of-freedom \gls{auv} model into a double-integrator system. The transformation enables us to develop the formation-control algorithm for simple kinematic point systems. Subsequently, through the design of specific path-following, formation-keeping, and collision-avoidance tasks, the fleet is controlled to follow a preplanned path in formation while avoiding collisions both within the fleet and with external obstacles. Because our reformulated \gls{nsb} method works directly with the second-order system given by the hand position controller, there is no need to transform desired velocities or accelerations into surge and orientation references, as has been done in previous works. Thus, the complexity level of the control design is reduced.

We review two different methods for compensating for unknown ocean currents. The stability of the system under both ocean current compensation methods is studied when the collision avoidance task is inactive. Two methods for external obstacle avoidance are developed for the double-integrator formulation. The first enables the fleet to avoid obstacles as one unit, keeping formation throughout the avoidance maneuver. In the second method, the vehicles are allowed to break formation in order to avoid collisions.

The \gls{nsb} method is inherently centralized, meaning there must be a central node that communicates and coordinates with all the vehicles in the fleet. We develop a novel decentralized reformulation of the \gls{nsb} algorithm. The key insight lies in reformulating the formation-keeping task as a consensus algorithm. Our resulting distributed method requires inter-vehicle communication of a path progress parameter as well as measurements of relative positions and velocities with neighboring vehicles. The proposed control law uses a sliding mode term in the formation-keeping sub-controller to eliminate formation-keeping errors introduced by the path-following task under non-complete communication graphs.

The closed-loop system is analyzed under both the centralized and the decentralized control law. We demonstrate that the formation-keeping and path-following subsystems can be analyzed independently, and analyze the stability for each of them using Lyapunov theory. The system is shown to be \gls{usges} under the centralized controller, whereas the decentralized controller ensures that the trajectories of the system remain bounded when the sliding-mode controller is approximated with a saturation function. With the ideal switching controller, the system is shown to be asymptotically stable.

We provide an extensive set of MATLAB simulation studies, in which both the centralized and decentralized \gls{nsb} methods are tested under different scenarios. The centralized method is tested with two different obstacle avoidance approaches. The decentralized method is first simulated in a general scenario and then compared to two other existing formation path-following methods from the literature. The MATLAB code for one of the external works was handed to us, while we implemented the other controller ourselves following details in the paper.

\section{Outline}\label{sec:outline}
The report is organized as follows. In Chapter~\ref{cha:vehicle_model}, the mathematical \gls{auv} model is presented, as well as all necessary assumptions. Chapter~\ref{cha:hand_position} presents the 3D hand-position input-output linearization method which transforms the highly non-linear equations of motion of the underactuated vehicles into a double-integrator system in position. Chapter~\ref{cha:formation_path_following} introduces notation, assumptions, and mathematical tools for formation path following of fleets of \glspl{auv}. The chapter introduces the mathematical description of a path as a continuous, differentiable parametric function, and it defines the fleet formation as a set of reference vectors relative to the barycenter of the fleet. The chapter also introduces some simple graph theory, which serves as a basis for the development of the distributed \gls{nsb} method.  In Chapter~\ref{cha:nsp}, background theory for the \gls{nsb} algorithm is presented before the general \gls{nsb} method for second-order systems is developed as a natural extension of the first-order \gls{nsb} method. Chapter~\ref{cha:nsb_tasks} presents the centralized \gls{nsb} method. Collision-avoidance, formation-keeping, and path-following tasks are developed for the second-order formulation so that the task solution provides desired accelerations. Then, modifications to the tasks that enable the fleet to compensate for unknown ocean currents and avoid external obstacles are presented. In Chapter~\ref{cha:closed_loop}, the closed loop properties of the centralized control algorithm presented in Chapter~\ref{cha:nsb_tasks} are studied. In Chapter~\ref{cha:distributed_NSB}, the \gls{nsb} method is modified to work in a distributed setting. The closed-loop stability analysis for this method is provided in Chapter~\ref{cha:distributed_closed_loop}. Both the centralized and decentralized control algorithms are simulated in various configurations in MATLAB \citep{matlab_2022} in Chapter~\ref{cha:sim_matlab}.  Finally, Chapter~\ref{cha:conclusions} presents conclusions and future work.
