%!TEX root = ../Thesis.tex
\chapter{Conclusions and Future Work}\label{cha:conclusions}
%
In this thesis, we have addressed the formation path-following problem with \glspl{auv} and proposed a novel control method based on the extended second-order \gls{nsb} algorithm. Our work makes several contributions to the field of autonomous fleet coordination and control. In particular, we have submitted a conference paper to the 62nd IEEE Conference on Decision and Control \citep{lie_formation_2023}.

Building upon the existing \gls{nsb} algorithm, we developed an extended second-order \gls{nsb} method that directly handles the double-integrator dynamics of \glspl{auv}. By leveraging the hand-position input-output linearizing controller, we express the entire fleet dynamics in task space resulting in a method without hidden low-level dynamics. The method enables us to express the task dynamics as interpretable spring-damper systems and eliminates the intermediate step of transforming desired velocities or accelerations into surge and orientation references, as has been done in previous works.\looseness=-1

To validate the performance and stability of our control method, we conducted a thorough closed-loop stability analysis of the joint formation-control and path-following system. The formation-keeping subsystem was shown to be \gls{ugas} and the path-following subsystem was shown to be \gls{usges}. 

One significant contribution of our research is the development of a novel approach to reformulating the centralized \gls{nsb} algorithm into a decentralized method. By treating formation keeping as a consensus problem we devised a distributed \gls{nsb} method that leverages inter-vehicle communication of path progress, relative positions, and relative velocities. Alternatively, with the right sensor set, the relative positions and velocities can be directly measured instead of communicated. This approach enables the application of the \gls{nsb} method in real-world systems subject to communication constraints, as the need for continuous communication with a central node is eliminated.\looseness=-1

Finally, extensive MATLAB simulation studies were conducted to evaluate the performance of both the centralized and decentralized NSB methods. Our simulations demonstrated the effectiveness of our approach and showcased that under ideal conditions it might perform better than existing methods. We evaluated the methods under various scenarios, including different choices for the external collision avoidance method. Although our method seemingly performed better than the first-order \gls{nsb} method, we noted the advantages of the first-order method, in particular, that it leverages robust, well-tested low-level maneuvering controllers. The hand-position controller used in our method is not as well-tested and studied in real-world applications.\looseness=-1

Future work includes testing the method on high-fidelity simulators such as \gls{dune} \citep{dune} and in real-world experiments. Testing the method's robustness in more realistic and complex underwater scenarios is a critical step before the method can be deployed in the real world. The use of high-fidelity simulators can provide a more accurate representation of the dynamics and environmental conditions that \glspl{auv} encounter in real-world operations. In particular, imperfect state measurements and actuation may significantly impact the underlying feedback-linearizing hand-position controller. 

Initial attempts were made to implement the method in \gls{dune}, but our efforts were not successful in robustly implementing the low-level hand-position controller. As the hand-position controller is a prerequisite for implementing the second-order \gls{nsb} method, implementation in \gls{dune} was not pursued further. Future work may therefore also include deeper research into the hand-position controller or alternative approaches to transforming the nonlinear equations of motion into double-integrator systems.\looseness=-1

